\documentclass{article}

\usepackage{styles}
\usepackage{titlesec}

\PaperTitle{Analyse de séries temporelles avec R}
\begin{document}
\maketitle


\hline
\vspace{0.5cm}
\noindent{\textbf{Formateur: Alexis Gabadinho}} \\
\textbf{Email: \href{mailto:alexis.gabadinho@protonmail.com}{alexis.gabadinho@protonmail.com}} \\
%\textbf{Office: Office} \\
% \multicolumn{3}{c}{Office Hours: TBD}\\
\hline                                   
\vspace{0.5cm}

\section*{Notes}

\begin{itemize}[noitemsep]
\item La formation se déroule sur deux journées
\item Environ 10 participants
\item Exemples / exercices pratiques faits sur les données de comptabilité nationale, en particulier le PIB
\item Les participants suivent par ailleurs une formation à R
\end{itemize}

\section{Environnement de travail (rappels)}

\begin{itemize}[noitemsep]
\item Utilisation de R et RStudio, R markdown
\item Importation de données à partir de fichiers \texttt{.csv} et \texttt{Excel}
\item Le \texttt{tidyverse} (utilisation de \texttt{dplyr}, objets \texttt{tibble})
\item Utilisation de \texttt{ggplot}
\end{itemize}

\section{Librairies spécialisées et structures de séries temporelles dans R}

\begin{itemize}[noitemsep]
\item Les objets \texttt{ts} et \texttt{tsibble}
\item Les librairies \textbf{forecast}, \textbf{feasts}, \textbf{fable}, ...
\end{itemize}

\section{Séries temporelles: définitions et propriétés}

\begin{itemize}[noitemsep]
\item Moyenne
\item Dispersion 
\item Autocorrélation
\item Stationarité
\item Saisonalité
\item Hétéroscédasticité
\end{itemize}


\section{Analyse descriptive et représentations graphiques d’une série temporelle}

\begin{itemize}[noitemsep]
\item Chronogramme
\item Lag plot
\item Corrélogramme (ACF)
\end{itemize}

\section{Régression linéaire}

\begin{itemize}[noitemsep]
\item Principes de la régression linéaire
\item Hypothèses et analyse des résidus: 
\begin{itemize}[noitemsep]
\item Normalité
\item Autocorrélation des résidus (test de Durbin-Watson)
\item Homoscédasticité (test de  Breush-Pagan)
\end{itemize}
\end{itemize}


\section{Transformation des données et stabilisation de la variance}

\begin{itemize}[noitemsep]
\item Transformation logarithmique
\item Désaisonnalisation par la régression linéaire 
\item Désaisonnalisation par les moyennes mobiles
\end{itemize}


\section{Décomposition d’une série temporelle}

\begin{itemize}[noitemsep]
\item Modèle additif
\item Modèle multiplicatif
\end{itemize}


\section{Prévision avec les méthodes de lissage exponentiel}

\begin{itemize}[noitemsep]
\item Lissage exponentiel simple
\item Lissage exponentiel double
\item Méthode de Holt-Winters
\end{itemize}

\section{Modélisation de séries stationnaires et non stationnaires}

\begin{itemize}[noitemsep]
\item Modèle ARMA
\item Modèle ARIMA
\end{itemize}


\section{Modèles multivariés}

\begin{itemize}[noitemsep]
\item Modèle VAR
\item Modèle à correction d'erreur (VEC)
\end{itemize}

\end{document}
